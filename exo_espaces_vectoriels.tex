\documentclass[twocolumn, landscape, french]{article}
\usepackage[T1]{fontenc}
\usepackage[utf8]{inputenc}
\usepackage{amsmath}
\usepackage{amssymb}
\usepackage{enumitem}
\usepackage{lmodern}
\usepackage[a4paper]{geometry}
\usepackage{babel}

\geometry{margin=1.5cm}
\setlength{\columnsep}{30pt}

\newcounter{exercice}
\newenvironment{exo}{
	\refstepcounter{exercice}
	\textbf{Exercice \theexercice.}
	\quad
}{
	\medskip
}	

\newcommand{\R}{\mathbf{R}}
\newcommand{\C}{\mathbf{C}}
\newcommand{\F}{\mathcal{F}}
\newcommand{\K}{\mathbf{K}}
\newcommand{\Q}{\mathbf{Q}}
\newcommand{\N}{\mathbf{N}}
\newcommand{\Z}{\mathbf{Z}}
\newcommand{\Vect}[1]{\mathrm{Vect}\left( #1\right)}
\renewcommand{\dim}[1]{\mathrm{dim}\left( #1\right)}

\begin{document}
	
	\section*{Questions de cours.}
	On note $\K = \R$ ou $\C$.
	\begin{enumerate}
		\item Soit $E$ un $\K$-espace vectoriel. Soit $\F = (e_1, \dots e_n)$ une famille de vecteurs de $E$. Définition de $\Vect{\F}$. Montrer que c'est le plus petit sous espace vectoriel de $E$ contenant $\F$. Majorer la dimension de $\Vect{\F}$. Donner une CNS d'égalité.
		\item Soit $E$ un $\K$-espace vectoriel. Soient $F$ et $G$ deux sous espaces vectoriels de $E$. Montrer que $F+G$ est un sous espace vectoriel, et que c'est le plus petit sous espace vectoriel contenant $F$ et $G$. Dans le cas où $F$ et $G$ sont de dimension finie, a-t-on en général $\dim{F+G}=\dim{F}+\dim{G}$ ? Donner une condition suffisante pour que $\dim{F+G}=\dim{F}+\dim{G}$  soit réalisée.
		\item Soit $n$ un entier naturel supérieur à 1. Montrer que $\mathcal{S}_n(\K)$ et $\mathcal{A}_n(\K)$ sont des sous espaces vectoriels de $\mathfrak{M}_n(\K)$ et qu'ils sont supplémentaires.
		\item Montrer que $A$, l'ensemble des fonctions impaires de $\R^\R$ et $S$ l'ensemble des fonctions paires de $\R^\R$ sont des sous espaces vectoriels de $\R^\R$ et qu'ils sont supplémentaires.
		\item Montrer qu'une famille de polynômes de degrés échelonnés ne contenant pas le polynôme nul est libre dans $\K[X]$. Que dire dune famille de $n+1$ polynômes de degrés échelonnés ne contenant pas le polynôme nul dans $\K_{n}[X]$ ?
		\item Démonstration de la formule de Taylor pour les polynômes.
	\end{enumerate}
	
	\section*{Exercices.}
	
	\begin{exo}
		Soit $E$ un $\K$-ev. Soient $F$ et $G$ deux sev de $E$. Montrer qu'il a équivalence entre
		\begin{enumerate}[label=(\roman*)]
			\item $F \cup G$ est un sous espace vectoriel
			\item $F \subseteq G$ ou $G \subseteq F$
		\end{enumerate}
	\end{exo}
	
	\begin{exo}
		Soit $E$ un $\K$-ev de dimension finie $n$. On considère deux sev de $E$ : $F$ et $G$. Le but de ce problème est de montrer qu'il y a équivalence entre
		\begin{enumerate}[label=(\roman*)]
			\item $F$ et $G$ admettent un supplémentaire commun
			\item $\dim{F}=\dim{G}$
		\end{enumerate}
		\begin{enumerate}
			\item Montrer que $(\mathrm{i}) \Rightarrow (\mathrm{ii})$.
			\item On suppose $(\mathrm{ii})$ et que $F\subset G$ ou $G\subset F$. Montrer que $F$ et $G$ admettent un supplémentaire commun. On suppose désormais que $F \nsubseteq G$ et $G \nsubseteq F$.
			\item  Pour montrer $(\mathrm{ii}) \Rightarrow (\mathrm{i})$ dans ce cas, on va procéder par récurrence descendante. Montrer que $(\mathrm{ii}) \Rightarrow (\mathrm{i})$ lorsque $\dim{F}=\dim{G}=n$.
			\item On note $p=\dim{F}=\dim{G} \in \left\lbrace 0, 1, 2, \dots,  n-1\right\rbrace $. On suppose que $(\mathrm{ii}) \Rightarrow (\mathrm{i})$ pour tous sev $H_1$ et $H_2$ de dimension $p+1$. Montrer qu'il existe $(x, y)$ in $F \times G$ tels que \[ G \oplus \Vect{x} \quad \text{ et }\quad F \oplus \Vect{y} \] 
			\item En déduire que ces deux SEV admettent un supplémentaire commun.
			\item Montrer que $F$ et $G$ admettent un supplémentaire commun de la forme : \[S \oplus \dots \]
			\item Conclure.
		\end{enumerate}
	\end{exo}
	
	\begin{exo}
		Soient $P$ un polynôme à coefficients réels et $a \in \R$. On suppose que 
		\[P(a) >0 \quad \text{ et } \quad \forall k \in \N^\star P^{(k)}(a) \geq 0\]
		Montrer que $P$ n'a pas de racines dans $\left[ a, \infty \right[$
	\end{exo}
	
	\begin{exo}
		On se propose de montrer que toute matrice nilpotente de $\mathfrak{M}_n(\K)$ est d'indice de nilpotence au plus $n$. Soit $N$ une matrice nilpotente. On note $p$ le plus petit entier tel que $N^p = 0$. Raisonnons par l'absurde et supposons $p > n$
		\begin{enumerate}
			\item Montrer qu'il existe $X \in \mathfrak{M}_{n,1}(\K)$ non nul tel que $N^{p-1}X \neq 0$. En déduire que $N^{k}X \neq 0$ pour tout $k \in \left\lbrace 0, 1, 2, \dots,  p-1\right\rbrace$
			\item Montrer que $(X, NX, N^2X, \dots, N^{p-1}X)$ est une famille libre de $\mathfrak{M}_{n,1}(\K)$. Conclure.
		\end{enumerate}
	\end{exo}
	
	\begin{exo}\textit{Polynômes de Newton.\\}
		On considère la suite de polynômes $(P_n) $ définie par
		\[\begin{cases}
			P_0 = 1\\
			\forall n \in \N : P_{n+1} = \frac{X - n}{n+1}P_n
		\end{cases}\]
	\end{exo}
	\begin{enumerate}[label=(\alph*)]
		\item Calculer les premiers termes de cette suite. Montrer que $(P_0, \dots P_n)$ est une base de $\K_n[X]$ pour tout $n \in \N$.
		\item On fixe un entier $n \geq 1$. Soit $Q$ un polynôme de  $\K_n[X]$. Déterminer les coordonnées de $Q$ dans cette base.
		\item Montrer que $Q$ est à coordonnées entières (dans cette base) si et seulement si $P(\Z) \subset \Z$.
	\end{enumerate}
	
	\newpage
	
	\begin{exo}
		On note $E$ l'espace vectoriel des suites réelles convergente.
		\begin{enumerate}[label=(\alph*)]
			\item Montrer que $\left\lbrace (x_n) \in E : x_n \longrightarrow 0\right\rbrace $ est un sous espace vectoriel de $E$. On le notera $F$.
			\item Déterminer un supplémentaire de $F$ dans $E$.
		\end{enumerate}
	\end{exo}
	
	\begin{exo}
		Soit $E$ un $\K$-ev. On se donne $\F_1$ et $\F_2$ deux familles finies de vecteurs de $E$. On note $V=\Vect{\F_1}$ et $W = \Vect{F_2}$ Montrer que \[V + W = \Vect{\F_1 \cup \F_2}\]
	\end{exo}
	
	\begin{exo}
		On rappelle qu'en dimension finie, un hyperplan d'un espace de dimension $d$ est un sous espace de dimension $d-1$.
		Soient $E$ un $\K$-ev de dimension finie, $F$ un sev de $E$ et $H$ un hyperplan de $E$. On suppose que $F \nsubseteq H$. Montrer que \[\dim{F \cap H} = \dim{F} - 1\]
	\end{exo}
	
	\begin{exo}
		Dans ce problème, on propose de montrer un résultat classique :  \begin{center}
			Tout hyperplan de $\mathfrak{M}_n(\K)$ contient une matrice inversible.
		\end{center}
		On rappelle qu'en dimension finie, un hyperplan d'un espace de dimension $d$ est un sous espace de dimension $d-1$.
		On fixe $H$ un hyperplan de $\mathfrak{M}_n(\K)$. On suppose que $I_n \notin H$ (sinon, $I_n$ étant inversible, il n'y a rien à montrer)
		\begin{enumerate}[label=(\alph*)]
			\item Montrer que $\Vect{I_n}$ est un supplémentaire de $H$. 
			\item Construisez une matrice inversible de $H$. Conclure. \textit{Indication :} On pourra s'intéresser aux matrices élémentaires $E_{i,j}$.
		\end{enumerate}
	\end{exo}
	
	\begin{exo}
		On note $E$ le $\R$ espace vectoriel des polynômes de degré au plus $n$. Pour $i \in \left\lbrace 0, 1, \dots, n\right\rbrace$ on note : \[F_i = \left\lbrace P \in E : \forall  j \leq n,  j \neq i, P(j) = 0 \right\rbrace \]
		Montrer que $F_i$ est un sous espace vectoriel de $E$ pour tout $i$ et que \[E = \bigoplus_{i=0}^nF_i\]
	\end{exo}
	
	\begin{exo}
		On note $E$ le sous espace vectoriel des fonctions de classes $\mathcal{C}^1$ sur $\R$. On pose \[ F = \left\lbrace f \in E \text{ tel que }f(0) = f'(0) = 0\right\rbrace \]
		Montrer qu'il s'agit d'un SEV de $E$. Cherchez un supplémentaire de $F$.
	\end{exo}
	
	\begin{exo}
		Soit $E$ un $\K$-ev de dimension finie $n$. On considère deux hyperplans de $E$ :$H_1$ et $H_2$. Montrer que ces deux hyperplans admettent un supplémentaire commun. Le résultat reste t-il vrai si $E$ est de dimension infinie ?
	\end{exo}
	
\end{document}